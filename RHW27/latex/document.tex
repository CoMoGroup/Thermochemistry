\documentclass[11pt]{article}
\title{Reaction Scheme}
\date{\today}
 \usepackage[pdftex]{graphicx}
 \graphicspath{{../structures/}}
 
 \usepackage{longtable}
 \usepackage{color}
% \definecolor{navy}{rgb}{0,0,0.5}

 
 %%% FONT FAFFAGE
  % Palatino for rm and math | Helvetica for ss | Courier for tt
\usepackage{mathpazo} % math & rm
\linespread{1.05}       % Palatino needs more leading (space between lines)
% above \linespread  conflicts with the setspace package, so instead we use
%\SetSinglespace{1.05} % for Palatino
\usepackage[scaled]{helvet} % ss (sans-serif)
% \usepackage[scaled=0.8]{couriers} % tt (typewriter)
\normalfont	
\usepackage[T1]{fontenc}
% \usepackage[scaled]{ulgothic} % 
\usepackage[scaled=0.8]{beramono} % http://www.tug.dk/FontCatalogue/beramono/

\begin{document}
%  \maketitle 
%  Table \ref{table-reaction-scheme} shows a reaction scheme generated by the Python script.  \newline
  % This is a comment, it is not shown in the final output.
  % The following shows a little of the typesetting power of LaTeX

%\begin{table}[htdp]
%\caption{CBS-QB3 energies}
%\label{table-reaction-scheme}
%\begin{center}
%\begin{tabular}{ l c @{ $\rightleftharpoons$ } c  c  c }
%\hline
%& reactants & products & 
%$\Delta E_{\rm rxn}$ &
%barrier \\
%\multicolumn{3}{c}{} & \multicolumn{2}{c}{(kcal/mol)} \\
%\hline

\begin{center}
\begin{longtable}{  l c @{ $\rightleftharpoons$ } c  c  c c }
\caption{CBS-QB3 Energies} \label{cbsqb3} \\
%This is the header for the first page of the table...
\hline \hline \\[-2ex]
& reactants & products & 
$\Delta E_{\rm rxn}$ &
\multicolumn{2}{c}{barrier} \\
\multicolumn{3}{c}{} & \multicolumn{3}{c}{(kcal/mol)} 
\\[0.5ex] \hline
   \\[-1.8ex]
\endfirsthead

%This is the header for the remaining page(s) of the table...
% \multicolumn{3}{c}{{\tablename} \thetable{} -- Continued} \\[0.5ex]
  \hline \hline \\[-2ex]
\multicolumn{3}{l}{CBS-QB3 Energies (kcal/mol)} & $\Delta E_{\rm rxn}$ &
\multicolumn{2}{c}{barrier}
\\[0.5ex] \hline
  \\[-1.8ex]
\endhead

%This is the footer for all pages except the last page of the table...
%  \multicolumn{3}{l}{{Continued on Next Page\ldots}} \\
\endfoot

%This is the footer for the last page of the table...
  \\[-1.8ex] \hline \hline
\endlastfoot

%Now the data...

%!TEX root =  Document.tex
\multicolumn{5}{p{\textwidth}}{That was a test. HJ should have an enthlapy of: 217.998 kJ/mol or 52.102 kCal/mol
} \\ 
\multicolumn{5}{p{\textwidth}}{That was totally anisogyric, but was a test. CH3OOH should have an H of -131 kJ/mol
} \\ 



\end{longtable}
\end{center}


% \hline
%\end{tabular}
%\end{center}
%\label{default}
%\end{table}%

\end{document}